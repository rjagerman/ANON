\documentclass[journal]{IEEEtran}

% Some very useful LaTeX packages include:
% (uncomment the ones you want to load, see original template for descriptions)
%\usepackage{ifpdf}
%\usepackage{cite}
%\usepackage[cmex10]{amsmath}
%\usepackage{algorithmic}
%\usepackage{array}
%\usepackage{fixltx2e}
%\usepackage{stfloats}
%\usepackage{dblfloatfix}
%\usepackage{url}

% *** Do not adjust lengths that control margins, column widths, etc. ***
% *** Do not use packages that alter fonts (such as pslatex).         ***
% There should be no need to do such things with IEEEtran.cls V1.6 and later.
% (Unless specifically asked to do so by the journal or conference you plan
% to submit to, of course. )

% correct bad hyphenation here
\hyphenation{op-tical net-works semi-conduc-tor}

% *** UTF8 ***
\usepackage[utf8]{inputenc}


\begin{document}

\bibliographystyle{plain}

% Title of the paper
\title{30 years of privacy research for the masses \& \newline 15 years of decentralization attempts}
% Authors
\author{Rolf Jagerman, Wendo Sab\'ee, Laurens Versluis, Martijn de Vos}

% The paper headers
%\markboth{Journal of \LaTeX\ Class Files,~Vol.~11, No.~4, December~2012}%
%{Shell \MakeLowercase{\textit{et al.}}: Bare Demo of IEEEtran.cls for Journals}
% The only time the second header will appear is for the odd numbered pages
% after the title page when using the twoside option.

% make the title area
\maketitle

% As a general rule, do not put math, special symbols or citations
% in the abstract or keywords.
\begin{abstract}
	The abstract goes here. Lorem ipsum dolor sit amet, consectetur adipiscing elit. Nunc et elit nec metus ultrices faucibus. Mauris volutpat libero eu arcu sodales, ac bibendum neque pulvinar. Sed et ipsum nunc. Quisque ac sem ut nunc tincidunt commodo. Cras at odio nibh. Pellentesque sodales ut tortor sed dictum. Cum sociis natoque penatibus et magnis dis parturient montes, nascetur ridiculus mus. 
\end{abstract}


\section{Introduction}
	% The very first letter is a 2 line initial drop letter followed
	% by the rest of the first word in caps.
	% 
	% form to use if the first word consists of a single letter:
	% \IEEEPARstart{A}{demo} file is ....
	% 
	% form to use if you need the single drop letter followed by
	% normal text (unknown if ever used by IEEE):
	% \IEEEPARstart{A}{}demo file is ....
	% 
	% Some journals put the first two words in caps:
	% \IEEEPARstart{T}{his demo} file is ....
	% 
	% Here we have the typical use of a "T" for an initial drop letter
	% and "HIS" in caps to complete the first word.
	% You must have at least 2 lines in the paragraph with the drop letter
	% (should never be an issue)
	\IEEEPARstart{W}{ith} the introduction and commercialization of the internet in 1995, it has become an important tool for anyone to freely and easily communicate. In many cases however, internet usage is either censored or monitored by third parties. In recent years the need for anonymous internet communication has become more apparent. Revealing government wrongdoings, such as in the case of WikiLeaks, or the ability to communicate freely in a censored environment are two major examples of why anonymous communication is important.
	
	Although much research has been done about anonymous internet communication, only few systems have been actually implemented and used in practice. One of the most important factors that impact anonymity in a communication system is the number of users. A sufficiently large numbers of users are required for the system to make guarantees about its ability to conceal and hide users' communication. This makes it difficult to introduce a new anonymous internet system, due to the initial lack of users. Another important factor is for the system to be decentralized. Whenever a network has centralized components and a malicious third party gains control over these components, the functionality and perhaps even the anonymity of the system is compromised.
	
	The most widely used anonymous communication system is TOR. In this paper we will analyze TOR and its semi centralized nature. Furthermore we will look at attempts that have been made to decentralize TOR as well as several other decentralized systems.
	
	This paper is structured as following: In section \ref{sec:tor} we will give an introduction and overview about Tor. After that we will talk about decentralization and its problems in section \ref{sec:problems}. Vulnerabilities in Tor are discussed in section \ref{sec:attacks}. Afterwards, we will discuss the current state of decentralized internet systems in section \ref{sec:decentralized}. A comparison of existing decentralized networks is made in section \ref{sec:comparison}. Finally we will conclude and discuss our findings in section \ref{sec:conclusion}.
	
\section{Introduction to Tor}
	\label{sec:tor}
	The implementation of The Onion Router was first described by the U.S. Navy Research Laboratory, as means to protect government communications from digital, as well as physical attacks, by hiding the location of the communicating party or parties \cite{goldschlag1996hiding}.

	In 2002, it was decided to ditch the old code base and reimplement the project as Tor, the Second Generation Onion Router, introducing perfect forward secrecy, directory servers, hidden services and more \cite{dingledine2004tor}.
		
	\subsection{Onion routing}
		As described in the original design paper of The Onion Router, network traffic is forwarded through a circuit of nodes, where each node only knows the previous and next node in the circuit. With a sufficiently long circuit of (independent) nodes, this means that two communicating parties can remain oblivious of each others physical location.
					
		Say we have a circuit consisting of four nodes: our trusted client node ($C$), an (entry) relay node ($X$), a (middle) relay node ($Y$) and an exit node ($Z$). In this case, there are two relay nodes, but this is not a fixed number (and could be increased for more security). Each node has its own public key and a corresponding private key. When building the circuit, our client generates a distinct secret for each of these nodes.
					
		The payload of each packet flowing through the circuit is first encrypted with the distinct secret for last node $Z$, then with the distinct secret for node $Y$, and last with the secret for node $X$. With each layer of encryption, a header is added with the address of the next node in the circuit, plus the used distinct secret encrypted with corresponding nodes public key. 
					
		After node $X$ receives this packet from out trusted node $C$, it decrypts the attached secret with its private key, and uses that secret to decrypt the rest of the packet. The result is a header with the address for node $Y$ and the payload encrypted with secrets for the following nodes in the circuit, which is forwarded to node $Y$.
					
		As a node receives a packet from the previous node, it peels off another layer of encryption, much as how you can peel an onion layer for layer, and forwards it to the next node in the circuit (as specified in the decrypted header). When exit node $Z$ decrypts the last layer, it forwards the unencrypted payload outside the network to the original destination that our trusted client $C$ tried to contact, acting as a traditional proxy.
					
		When exit node $Z$ receives a response, this whole process is applied in reverse order, encrypting the payload with its secret along the way, instead of decrypting. When our client $C$ receives the packet, it peels off all the encryption layers to retrieve the unencrypted payload.
		
		With the second generation onion routing used in Tor, a modified algorithm is used to derive the encryption keys, called \emph{telescoping path-building}, which also provides perfect forward secrecy. This algorithm is described in section \ref{ss:tor_circuit}.
				
	\subsection{Directory servers}
		The original Onion Router implementation used in-band status updates where each node broadcasts know nodes to its neighbours. An attacker could exploit this to isolate and limit the knowledge of a client, forcing connections through malicious nodes. Beside possible security disadvantages, in-band status updates also have the disadvantage that it takes longer to propagate throughout the network and create a global consensus.
					
		To mitigate these concerns, Directory Servers were introduced to Tor during its reimplementation to keep a redundant central consensus about the network. They act as HTTP servers to which Tor nodes can publish signed information about themselves. Tor clients can in turn download this information.
					
		This information is signed by the Directory Servers before it is distributed to the Tor clients. The keys to verify these signatures are preloaded in the Tor software, along with the list of Directory Servers, which implies trust by the Tor client in the Directory Servers.

	\subsection{Relay and exit nodes}
		The Tor network consists of several components. The clients in the Tor network are known as Onion Proxies. The software to run an Onion Proxy is available for free on the Tor website \cite{torprojectwebsite} and is easy to configure. The Onion Proxies are responsible for downloading the directory information, establish circuits across the network and handle connections from user applications.
		
		The routing in the network is done by Onion Routers, also called relay nodes. The relay nodes relays the data from the Onion Proxy to the web server across a circuit (circuits are described in 2D). Each Onion Router is connected to every other Onion Router with a TLS connection \cite{tlsprotocol}. Each circuit has three type of Onion Routers \cite{mccoy2008shining}:
		\begin{itemize}
			\item{The entrance Tor router:} this router is directly connected to an Onion Proxy and can observe the origin of a request through the Tor network. The entrance router sends the packet to the middle Tor router.
			\item{The middle Tor router:} this router is connected to the entrance router and the exit router.
			\item{The exit Tor router:} this router is connected to the web server. Note that the exit Tor router is the only router that can observe the final destination of the request.
		\end{itemize}
		
		The first router in a circuit is the entrance router. The entrance router sends the data to one of the middle router which forwards the data to the exit router.
			
	\subsection{Circuit creation}
		\label{ss:tor_circuit}
		As described in the previous section, data on the Tor networks travels over circuits. The data travels over the circuit in fixed-size cells that are 512 bytes long. Each cell has a header and a payload. The header consists of a number that identifies the circuit of the cell and a command that indicates what to do with the cell’s payload.
		
		Before a circuit can be established, a path has to be chosen. The current path selection algorithm in Tor selects nodes based on the bandwidth of the nodes [link naar cacr2011-20]. Nodes that have more bandwidth, have a higher probability to be chosen for the circuit setup, however, the same node can't be used more than once in a circuit.
		
		Suppose Alice is an Onion Proxy that wants to connect through the Tor network to a web server. Circuit setup uses the Diffie-Hellman key-exchange protocol \cite{diffiehellman} to establish a shared secret between nodes. To create a new circuit, Alice first sends a \emph{create} cell with the first half of the Diffie-Hellman handshake ($ g^b $) to the first node in her selected path (for example, OR1). OR1 sends a \emph{created} cell back with the second half of the key ($ g^b $) along with a hash of the final key. Now both Alice and OR1 have a shared key they use to encrypt and decrypt data between them.
		
		So now Alice has a connection with the first Onion Router in the circuit. To extend the circuit to OR2, Alice first sends a \emph{relay extend} cell to OR1. This cell contains the address of the next Onion Router in the circuit and the first half of the key to use in the communication between her and OR2 ($ g^{a_2} $). OR1 takes this first half of the key and sends a \emph{create} cell with this key to OR2. When OR1 receives a created cell, OR1 passes this cell to Alice. Now Alice and OR2 share a common key: $ K = g^{a_2b_2} $. The same procedure can be used to extend the circuit with more nodes.
			
	\subsection{Disadvantages}
		While Tor guarantees anonymity, there are some disadvantages using it. The main disadvantage is that the Tor network is slow. According to the Tor Metrics project [link naar tor metrics project], it takes about 6 seconds to download 1 MiB of data.
		
		According to Dingledine et al \cite{dingledine2009performance}, there are six reasons why Tor is not optimal. In this section, we will summarize these reasons and explain what could be done to fix them.
		
		First of all, the congestion control does not work well. The network has some problems handling bulk transfers, such as when downloading large files or streaming high-quality videos. The congestion control could be improved by using an unreliable protocol for links between Tor relays.		
		
		Some Tor users put more traffic on the network than they contribute by running a Onion Router. This means that these users are slowing the network down as they use more traffic than giving to the network. A possible solution for this is to throttle certain protocols at exit nodes or at Onion Proxies.
		
		Also, the Tor network doesn't have the capacity to handle all the users that want privacy on the Internet. By increasing the amount of Onion Routing in the Tor network, the capacity is increased. Incentives such as LIRA \cite{jansen13lira} could make more users run a Onion Router, thus increasing the capacity of the network.		
		
		The current path selection algorithm of Tor doesn't distribute the load evenly over the network. The problem is that the current selection strategy is optimal when the network is fully loaded. Using a better path selection algorithm could increase the capacity of the network and the overall user experience.		
		
		Another problem is that the Tor clients are not optimal at handling latency and connection failures. For example, if extending a circuit fails, the entire circuit is abandoned. An improvement would be to first try to extend the circuit to some other places. If that fails, the circuit could be abandoned. Also, a better timeout mechanism could be chosen for building circuits.		
		
		Much of the overhead of the network is in downloading the directory information. There is also overhead in the TLS connection between the nodes in the network. Removing the empty TLS application record which could reduce the overhead in the TCP/IP header with 6.3\%.

\section{Problems when decentralizing}
		hier komt de introductie
		
		\subsection{Bootstrapping}
	
	\section{Current decentralized systems}
		Many other decentralized peer-to-peer networks have been proposed and some have been implemented [linkjes naar papers]. Examples of fully decentralized networks are Gnutella, Freenet, Freehaven, Tapestry, Kademlia and Tarzan. Each of these decentralized networks will be discussed in the next section. Note that we're only considering fully decentralized networks here. Networks that are partially or hybrid centralized, such as KaZaA and Napster, are not discussed here.[link naar survey p2p?]
	
		\subsection{Gnutella}
			Gnutella is a decentralized peer-to-peer network used for distributed search of files. Since the network is fully decentralized, peers in the network are called 'servents', a combination of servers and clients. Each peer can act both as a server, answering queries, or as a client, requesting and executing search queries. In order for a client to bootstrap, a new peer connects to one of several known hosts that are almost always available. These peers can be found on http://gnutellahosts.com. Once the peer has joined, there are several messages a servant can send out:
		
			\begin{itemize}
				\item A peer sends a PING message to it's neighbours to announce it's presents. This message is forwarded to other peers and each peer sends a PONG message back.
				\item The peer can issue QUERY answers. Other peers responds with a QUERY RESPONSE message to specify whether the file that was issued in the query, was found or not.
				\item To transfer items between peers, the GET and PUSH messages are being used.
			\end{itemize}
			
			Gnutella is an unstructured network which means that the placement of data items is not based on any knowledge of the topology nor the contents of the file. To search for a file, a flooding algorithm is used.
		
		\subsection{Freenet}
			In Freenet, each data item is represented by a key that is independent of the location of the file. Freenet is called a loosely structured network because of this. To issue a query, the request is passed from client to client where each client makes a decision about the location to send the request next.
		
			There are three types of file keys in Freenet: the first one is called the Keyword-Signed Key (KSK) which is derived from a short description of the file. Another key is the Signed-Subspace key which enables personal namespaces. This key contains a public key and a private key where the private key is used to store the data and the public key is used in the queries of the file. The third type of key is the Content-Hash Key (CHK) which is used for updating and splitting of contents.
		
			The routing algorithm for storing and retrieving data is dynamically and can adjust to the topology of the network. Each peer has only knowledge about his neighbours. Each request has a Hops-To-Live timer which indicates how many peers the request may traversal. Each peer decrements the timer by one and when the timer reaches zero, the request isn't forwarded any more. Results of queries are being cached in intermediate nodes to reduce the time for a query response.
			
		\subsection{NISAN}
			Information about NISAN.
		
		\subsection{Tribler}
			Information about Tribler.
		
		\subsection{Tapestry}
			Tapestry is based on the plaxton mesh data structure [linkje naar plaxton paper], which maintains pointers to node in the network whose IDs match the elements of a tree-like structure of ID prefixes up to a digit position. A property of Tapestry is that it offers load distribution and routing locality. Tapastry is very similar to pastry [link naar pastry paper] but there are some small differences in in the handling of network locality and data replication.
		
			Like Gnutella, peers can take the role of a client, issuing requests and the role of a server where objects are stored. A peer can also function as a router, which forwards an incoming message. The routing algorithm is based on the destination ID of the packet. Routers are using local routing maps to route messages to the destination ID digit by digit. The routing system ensures that each peer in the system can be found in a logarithmic amount of hops.
		
			Tapestry is a fundamental component of Oceanstore [linkje naar paper over Oceanstore], a decentralized storage system. Tapestry is also used in systems such as Bayeux and SpamWatch.
		
		\subsection{Tarzan}
			Information about Tarzan.
		
		\subsection{Other decentralized systems}
			Some other decentralized peer-to-peer systems has been developed, focussing on anonymity. An example of such a system is AP3 (Anonymizing Peer-to-Peer Proxy) [linkje naar AP3 paper] which makes cooperative, decentralized anonymous communication possible. The AP3 system provides clients with three primitives: anonymous message delivery, anonymous channels and secure pseudonyms. Users are able to send and receive unicast, multicast and anycast messages anonymously. The strategy that AP3 is using for message delivery is similar to that of Tarzan: it relies on a network of peers to forward messages. A node along the request path, does not know whether the node from which it receives a message is the message's originator or simply another forwarding peer.
			
			<hier moet nog een stukje over i2p komen voor degene die dat paper heeft gelezen>
		
	\section{Tor vulnerabilities and attacks}
		Besides the disadvantages mentioned in the previous section, Tor also suffers from several vulnerabilities that can be exploited trough attacks [cites?]. In this section we will summarize some of the most well known problems with Tor.

		\subsection{Browser based attacks}
			Traffic analysis can be used to attack the anonymity of a user (a Tor client) browsing the web using Tor \cite{abbott2007browser}. By misusing the exit policy of Tor one can reduce the time required to perform the analysis.
		
			By running a HTTP exit node and an Tor router that eventually will act as an entry node in the network, an adversary can discover the identity of a user. The exit node injects an invisible iframe containing some JavaScript into the web pages that pass by, each sending a unique ID to a malicious web server. Every ten minutes the Tor client chooses a new circuit and eventually an unlucky Tor client picks and uses the malicious entry node that was placed in the network.
		
			By performing traffic analysis to compare the unique IDs of the web server and the circuits passing trough the entry node, a user can be identified via matches. A similar system can be set up only using the HTML meta refresh tag. To increase the odds of a user choosing the malicious exit node, one can run the exit node on unpopular ports. There are usually only a few exit nodes running on file sharing ports 4661 to 4666.
		
			The solution for the JavaScript injection attack is disabling active content systems in the browser. For the HTML only variant one would have to use HTTPS to prevent man-in-the-middle attacks.
	
	\section{Comparison existing decentralized  networks}
		
		
		\subsection{Comparison and differences}
			Comparison table.
	
	\section{Conclusion}
		The conclusion goes here.

	\appendices
%		\section{Proof of the First Zonklar Equation}
%			Appendix one text goes here.
			
			% you can choose not to have a title for an appendix
			% if you want by leaving the argument blank
%		\section{}
%			Appendix two text goes here.
			
			
		% use section* for acknowledgement
%		\section*{Acknowledgment}
%			The authors would like to thank...
			
			
		% Can use something like this to put references on a page
		% by themselves when using endfloat and the captionsoff option.
%		\ifCLASSOPTIONcaptionsoff
%		  \newpage
%		\fi
			
		% trigger a \newpage just before the given reference
		% number - used to balance the columns on the last page
		% adjust value as needed - may need to be readjusted if
		% the document is modified later
		%\IEEEtriggeratref{8}
		% The "triggered" command can be changed if desired:
		%\IEEEtriggercmd{\enlargethispage{-5in}}
		
		% references section
		
		% can use a bibliography generated by BibTeX as a .bbl file
		% BibTeX documentation can be easily obtained at:
		% http://www.ctan.org/tex-archive/biblio/bibtex/contrib/doc/
		% The IEEEtran BibTeX style support page is at:
		% http://www.michaelshell.org/tex/ieeetran/bibtex/
		%\bibliographystyle{IEEEtran}
		% argument is your BibTeX string definitions and bibliography database(s)
		%\bibliography{IEEEabrv,../bib/paper}
		%
		% <OR> manually copy in the resultant .bbl file
		% set second argument of \begin to the number of references
		% (used to reserve space for the reference number labels box)
	\bibliography{paper_references}

\end{document}


