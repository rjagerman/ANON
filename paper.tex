
\documentclass[journal]{IEEEtran}

% Some very useful LaTeX packages include:
% (uncomment the ones you want to load, see original template for descriptions)
%\usepackage{ifpdf}
%\usepackage{cite}
%\usepackage[cmex10]{amsmath}
%\usepackage{algorithmic}
%\usepackage{array}
%\usepackage{fixltx2e}
%\usepackage{stfloats}
%\usepackage{dblfloatfix}
%\usepackage{url}

% *** Do not adjust lengths that control margins, column widths, etc. ***
% *** Do not use packages that alter fonts (such as pslatex).         ***
% There should be no need to do such things with IEEEtran.cls V1.6 and later.
% (Unless specifically asked to do so by the journal or conference you plan
% to submit to, of course. )


% correct bad hyphenation here
\hyphenation{op-tical net-works semi-conduc-tor}

% *** UTF8 ***
\usepackage[utf8]{inputenc}


\begin{document}

	% Title of the paper
	\title{ANON paper}
	% Authors
	\author{Rolf Jagerman, Wendo Sab\'ee, Laurens Versluis, Martijn de Vos}

	% The paper headers
	%\markboth{Journal of \LaTeX\ Class Files,~Vol.~11, No.~4, December~2012}%
	%{Shell \MakeLowercase{\textit{et al.}}: Bare Demo of IEEEtran.cls for Journals}
	% The only time the second header will appear is for the odd numbered pages
	% after the title page when using the twoside option.

	% make the title area
	\maketitle

	% As a general rule, do not put math, special symbols or citations
	% in the abstract or keywords.
	\begin{abstract}
		The abstract goes here. Lorem ipsum dolor sit amet, consectetur adipiscing elit. Nunc et elit nec metus ultrices faucibus. Mauris volutpat libero eu arcu sodales, ac bibendum neque pulvinar. Sed et ipsum nunc. Quisque ac sem ut nunc tincidunt commodo. Cras at odio nibh. Pellentesque sodales ut tortor sed dictum. Cum sociis natoque penatibus et magnis dis parturient montes, nascetur ridiculus mus. 
	\end{abstract}


	\section{Introduction}
		% The very first letter is a 2 line initial drop letter followed
		% by the rest of the first word in caps.
		% 
		% form to use if the first word consists of a single letter:
		% \IEEEPARstart{A}{demo} file is ....
		% 
		% form to use if you need the single drop letter followed by
		% normal text (unknown if ever used by IEEE):
		% \IEEEPARstart{A}{}demo file is ....
		% 
		% Some journals put the first two words in caps:
		% \IEEEPARstart{T}{his demo} file is ....
		% 
		% Here we have the typical use of a "T" for an initial drop letter
		% and "HIS" in caps to complete the first word.
		% You must have at least 2 lines in the paragraph with the drop letter
		% (should never be an issue)
		I wish you the best of success.
		
		
	\section{Tor design}
		
		This section describes 
		
		\subsection{Onion routing}
		
		\subsection{Directory servers}
		
		\subsection{Relay and exit nodes}
		
		\subsection{Circuit creation}
		
		\subsection{Disadvantages}
		
		
	\section{Conclusion}
		The conclusion goes here.
		
		
	\appendices
		\section{Proof of the First Zonklar Equation}
			Appendix one text goes here.
			
			% you can choose not to have a title for an appendix
			% if you want by leaving the argument blank
		\section{}
			Appendix two text goes here.
			
			
		% use section* for acknowledgement
		\section*{Acknowledgment}
			The authors would like to thank...
			
			
		% Can use something like this to put references on a page
		% by themselves when using endfloat and the captionsoff option.
		\ifCLASSOPTIONcaptionsoff
		  \newpage
		\fi
			
		% trigger a \newpage just before the given reference
		% number - used to balance the columns on the last page
		% adjust value as needed - may need to be readjusted if
		% the document is modified later
		%\IEEEtriggeratref{8}
		% The "triggered" command can be changed if desired:
		%\IEEEtriggercmd{\enlargethispage{-5in}}
		
		% references section
		
		% can use a bibliography generated by BibTeX as a .bbl file
		% BibTeX documentation can be easily obtained at:
		% http://www.ctan.org/tex-archive/biblio/bibtex/contrib/doc/
		% The IEEEtran BibTeX style support page is at:
		% http://www.michaelshell.org/tex/ieeetran/bibtex/
		%\bibliographystyle{IEEEtran}
		% argument is your BibTeX string definitions and bibliography database(s)
		%\bibliography{IEEEabrv,../bib/paper}
		%
		% <OR> manually copy in the resultant .bbl file
		% set second argument of \begin to the number of references
		% (used to reserve space for the reference number labels box)
	\begin{thebibliography}{1}
		
		\bibitem{IEEEhowto:kopka}
			H.~Kopka and P.~W. Daly, \emph{A Guide to \LaTeX}, 3rd~ed.\hskip 1em plus 0.5em minus 0.4em\relax Harlow, England: Addison-Wesley, 1999.
		
	\end{thebibliography}

\end{document}


