As we saw in section \ref{sec:tor}, we would like to have a decentralized anonymous network. This isn't an easy task: 15 years of decentralization attempts on the Tor network are behind us. Several well-used decentralized peer-to-peer overlay networks are active such as Gnutella [link], Freenet [link] and Tapestry. These networks will be described in more detail in the next chapter. Among these networks are some overlay anonymous networks such as Tarzan.

There are advantages when using a centralized authority: this authority can be used for bootstrapping, key management and to manage the reputation of the users. When removing this central authority, these components needs to be spread across the peers in the network. In this section we will explain the various problems that are involved when decentralizing Tor. In the next section, we will explain several decentralized peer-to-peer networks and make a comparison between them.

As we saw in the previous section, we would like to have a decentralized anonymous network. This isn't an easy task: 15 years of decentralization attempts on the Tor network are behind us. Several well-used decentralized peer-to-peer overlay networks are active such as Gnutella [link], Freenet [link] and Tapestry [link]. These networks will be described in more detail in the next chapter. Among these networks are some overlay anonymous networks such as Tarzan and Torks [link].

There are advantages when using a centralized authority: this authority can be used for bootstrapping, key management and to manage the reputation of the users. When removing this central authority, these components needs to be spread across the peers in the network. In this section we will explain the various problems that are involved when decentralizing Tor. In the next section, we will explain several decentralized peer-to-peer networks and make a comparison between them.
	
\subsection{Bootstrapping new nodes}
	If a Tor user decides to donate some of his bandwidth by running a bridge or a relay and thus creating a new node in the network, there has to be a starting point where this new node can discover neighbors in the network to connect with. In Tor, a directory server can tell the new node what his neighbors are and where to find them \cite{dingledine2004tor}.
	
	Moving this system to a non peer-to-peer base is difficult: Dingledine et al. stated that this indeed is still an open problem. With decentralized systems, there is no central directory server to tell a new node where to locate neighbors and there might be a lot of malicious nodes in the network. Some systems such as Tarzan, MorphMix and Pastry \cite{rowstron2001pastry, rennhard2002introducing} are decentralized but they do suffer from performance issues. Some of these issues can be found in the last section when comparing these systems.

\subsection{NAT traversal}
	A truly decentralized system requires the participating nodes to have direct connection to each other. Because of the limited availability of IPv4 network addresses, most consumer grade internet connections only provide one network address per subscriber, shared by all the devices connected to the subscribers network using Network Address Translation (NAT). With IPv4 network addresses getting more scarce, some Internet Service Providers even put more than one subscribers behind a single network address, using a carrier-grade NAT.
	
	A NAT-based system works by creating a local, private network which a NAT-enabled router connects to the internet. The local network uses network addresses from the private ranges (e.g. 10.0.0.0/8 or 192.168.0.0/16).
	
	When a local device sends a packet to the internet, the router replaces these private addresses, including the source ports, with its own public address before forwarding it to the internet. It saves these translations in a local table. Once it receives packets, it looks in this table and replaces the public network address and port with the associated private network address and port. If no corresponding entry exists in the routers table, the packet is dropped. This means that when contacting a device behind a NAT, there must be an existing entry in this table.
	
	There are several techniques to add an entry to the routers NAT table. Universal Plug and Play (UPnP) is one of those techniques, where the local device uses a HTTP request to the router to associate a port with the devices private network address. This technique is not available everywhere and sometimes considered a security risk. Different implementations of NAT require different techniques, such as \textit{hole punching}, \textit{relaying} or \textit{reversal}, as described by Wocker et al. [http://www.pap.vs.uni-due.de/files/wacker-nat-traversal.pdf].

\subsection{Incentives in decentralized systems}
	As described in section \ref{ss:tor_disadvantages}, people are usually using more bandwidth than that they donate. There are multiple proposals \cite{dingledine2010building, jansen13lira} to introduce incentives into Tor. If one would also have to build an incentive system into a decentralized system, they would have to find a way to manage the ratings of each client in the network, in such a way that they cannot be falsely modified. In other words, the reputation data has to be accurate and reliable. Besides the integrity of this data, the traffic it generates on the network should have minimal impact on the overall performance.
	
	Rahman \cite{rahman2009survey} proposes several options to build in incentives in a peer-to-peer network. The first proposal that is described is the so called \emph{Warm-glow Model}. This model determines the percentage of free-rides based on the probabilistic population distribution. If the percentage is above a certain threshold, the system will show signs of diminishing marginal returns.
	
	The second proposal is using monetary schemes. While it is not exemplified a lot, the main idea is to use a virtual currency as incentive. The problems with this approach are the scalability and the hidden costs of this service.
	
	The third proposal is \emph{Reciprocity-Based Schemes}. Using this approach, a peer maintains a behavior history of other peers in the network. These schemes can be based on two somewhat reciprocities: direct reciprocity or indirect reciprocity. The formal are more suitable for longer relationships between peers. The latter is more scalable but they rely on third party and must handle trust issues themselves.

\subsection{Key exchange}
	With a decentralized network, using a centralized authority for managing the keys, is not possible. This means that for secure communication, peers have to exchange the keys directly with each other, without a trusted party between them. Diffie-Hellman is a very popular algorithm for exchanging keys between two parties. Diffie-Hellman is for example used in the circuit setup in Onion Routing (see secion II-D), however, it is possible for an adversary to manipulate with the keys send between two parties, making the protocol vulnerable for a man-in-the-middle attack. This weakness makes it possible for an adversary to decrypt all message sent between the two parties.

	Tor currently uses an interactive forward-secret key-exchange protocol called the \emph{Tor Authentication Protocol} (TAP) [linkje naar ace]. This protocol uses telescoping which means that the initiator negotiates session keys with each successive hop in the circuit. There are several proposals for more efficient key exchange methods. One of them is \emph{ACE}, an one-way authenticated key exchange protocol. The authors of this methods claim to have an 46\% efficiency improvement on the side of the client and an efficiency improvement of nearly 19\% on the side of the Onion Routers. ACE requires clients to send one extra element in the key exchange. This however, does not introduce any overhead because the element fits in the unused space in a cell.